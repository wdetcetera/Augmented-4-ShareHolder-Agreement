\section{EXIT MECHANISMS}

\subsection{Exit Events} \label{subsec:ExitEvents}
The following constitute Exit Events:
\begin{enumerate}[label=(\alph*)]
\item Initial Public Offering (IPO);
\item Trade sale of all or substantially all Company shares or assets;
\item Merger or consolidation resulting in change of control;
\item Winding up of the Company.
\end{enumerate}

\subsection{IPO Provisions}
\begin{enumerate}[label=(\alph*)]
\item Decision to pursue IPO requires 75\% Shareholder approval.
\item All Shareholders must cooperate with IPO process including:
    \begin{enumerate}[label=(\roman*)]
    \item accepting reasonable escrow arrangements;
    \item providing all required information;
    \item executing necessary documentation;
    \item appointing recommended advisers for technology and IP due diligence.
    \end{enumerate}
\end{enumerate}

\subsection{Trade Sale}
\begin{enumerate}[label=(\alph*)]
\item Board may recommend trade sale to Shareholders.
\item Sale requires 75\% Shareholder approval.
\item Process must ensure:
    \begin{enumerate}[label=(\roman*)]
    \item competitive tension;
    \item protection of Company IP, particularly AI models and voice technology;
    \item fair value for Shareholders;
    \item appropriate warranties and indemnities related to technology performance.
    \end{enumerate}
\end{enumerate}

\subsection{Buy-Out Mechanism}
\begin{enumerate}[label=(\alph*)]
\item Any Shareholder may offer to buy all other Shareholders' Shares.
\item Offer must be:
    \begin{enumerate}[label=(\roman*)]
    \item in writing;
    \item at fair market value determined by an independent valuer with expertise in technology companies;
    \item on identical terms to all Shareholders;
    \item fully funded or subject to reasonable funding conditions.
    \end{enumerate}
\item Other Shareholders have 30 Business Days to make a counter-offer or accept.
\end{enumerate}

\subsection{Deadlock Provisions} \label{subsec:DeadlockResolution}
\begin{enumerate}[label=(\alph*)]
\item A ``Deadlock'' shall be deemed to exist if:
    \begin{enumerate}[label=(\roman*)]
    \item a resolution on a Reserved Matter requiring unanimous approval has been voted upon at two consecutive properly convened Board meetings or Shareholder meetings at least 14 days apart;
    \item the resolution has not been passed due to dissenting votes or abstentions of one or more Directors or Shareholders; and
    \item in the reasonable opinion of at least one Director or Shareholder, such failure to pass the resolution is materially detrimental to the Company's business.
    \end{enumerate}
\item If a Deadlock arises and cannot be resolved within 30 Business Days through good faith negotiations between the Shareholders, any Shareholder may serve a written notice (a ``Deadlock Notice'') on the other Shareholders and the Company, identifying the nature of the Deadlock.
\item \textbf{Cooling-Off Period and Structured Negotiation:} Upon receipt of a Deadlock Notice:
    \begin{enumerate}[label=(\roman*)]
    \item A mandatory cooling-off period of 20 Business Days shall commence (the ``Cooling-Off Period'').
    \item During the Cooling-Off Period, the Shareholders shall refrain from taking any material action related to the subject matter of the Deadlock that could exacerbate the situation or prejudice any party's position.
    \item Within 5 Business Days following the Deadlock Notice, the Shareholders shall meet to discuss the Deadlock in person or by video conference with the objective of understanding each party's position and concerns in detail.
    \item The Shareholders shall use this meeting to identify possible compromise solutions and establish common ground, documented in a shared memorandum.
    \item The Shareholders shall each designate a senior representative who has not been directly involved in the disputed matter (where possible) to lead follow-up discussions.
    \item These designated representatives shall meet at least twice during the Cooling-Off Period to explore compromise solutions.
    \end{enumerate}
\item \textbf{Structured Mediation Process:} If the Deadlock remains unresolved after the Cooling-Off Period:
    \begin{enumerate}[label=(\roman*)]
    \item The Shareholders shall jointly appoint an independent, qualified mediator with experience in commercial disputes in the technology sector from a list of pre-approved mediators maintained by the Australian Disputes Centre or a similar organization (the ``Mediator'').
    \item If the Shareholders cannot agree on a Mediator within 5 Business Days, each Shareholder shall nominate one mediator, and those nominees shall together select the Mediator within a further 5 Business Days.
    \item The mediation shall commence within 15 Business Days of the Mediator's appointment and follow a structured process including:
        \begin{itemize}
        \item Pre-mediation submissions to be provided to the Mediator at least 5 Business Days before the mediation;
        \item A joint opening session to set the agenda and process;
        \item Both joint and separate caucus sessions at the Mediator's discretion;
        \item Focused discussions on resolving the specific issue(s) causing the Deadlock; and
        \item Documented proposed solutions from each party.
        \end{itemize}
    \item The mediation shall continue for a period of up to 20 Business Days, unless extended by mutual agreement of the Shareholders.
    \item The costs of the mediation shall be borne equally by the Shareholders, regardless of the outcome.
    \item All discussions in the mediation shall be confidential and without prejudice.
    \end{enumerate}
    \item \textbf{Enhanced Mediation Commitment:} The Shareholders agree that:
    \begin{enumerate}[label=(\roman*)]
        \item They will participate in the mediation in good faith with a genuine commitment to explore compromises and solutions;
        \item Each Shareholder shall be represented in the mediation by a person with full authority to make binding decisions (or with immediate access to such authority);
        \item If the Mediator believes that additional sessions would be productive, the mediation period shall be extended for up to 10 additional Business Days;
        \item If the Mediator proposes a compromise solution that they believe is fair and reasonable, the Shareholders commit to seriously consider such proposal before rejecting it;
        \item If the Mediator determines that a particular Shareholder is not participating in good faith, they may document this in a report to be considered in any subsequent proceedings.
        \end{enumerate}
    \item \textbf{Third-Party Expert Facilitation:} If the mediation has not resolved the Deadlock:
    \begin{enumerate}[label=(\roman*)]
    \item Before proceeding to the Fair Valuation Buyout Option or Texas Shootout procedure, the Shareholders shall consider appointing an independent industry expert with relevant technical or commercial expertise.
    \item The expert shall provide a non-binding evaluation of the disputed matter, meet with the Board and key executives to understand the technical, commercial, or operational aspects of the dispute, prepare a written recommendation for resolving the Deadlock, and present recommendations at a special Board meeting.
        \item The appointment of such expert shall be by mutual agreement of the Shareholders, and the costs shall be borne equally.
        \item The process shall be completed within 30 Business Days of the expert's appointment.
        \item If no agreement is reached on appointing an expert, or if the expert's recommendation is not accepted by all Shareholders within 10 Business Days, then the Shareholders may proceed to the Fair Valuation Buyout Option.
        \end{enumerate}
\item \textbf{Fair Valuation Buyout Option:} If the Deadlock remains unresolved following the structured mediation process:
    \begin{enumerate}[label=(\roman*)]
    \item Before proceeding to the Texas Shootout procedure, any Shareholder (the ``Offering Shareholder'') may deliver a written notice (the ``Fair Value Buyout Notice'') offering to purchase all (but not less than all) of the Shares held by the other Shareholders (the ``Offeree Shareholders'').
    \item Upon delivery of a Fair Value Buyout Notice, an Independent Valuer shall be appointed to determine the Fair Market Value of the Shares.
    \item The Independent Valuer shall determine the Fair Market Value having regard to standard valuation methodologies appropriate for technology companies, the specific nature of the Company's AI voice agent business, historical financial performance and future projections, the value of intellectual property and technology, customer relationships and contracts, and market conditions.
    \item The Offeree Shareholders shall have 15 Business Days from receipt of the valuation report to accept or decline the offer.
    \item If declined, the Offeree Shareholders may deliver a counter Fair Value Buyout Notice within 10 Business Days.
    \end{enumerate}
\item \textbf{Texas Shootout Procedure:} If the Deadlock is not resolved through previous procedures, any Shareholder (the ``Initiating Shareholder'') may trigger the Texas Shootout procedure:
    \begin{enumerate}[label=(\roman*)]
    \item The Initiating Shareholder shall deliver a notice (the ``Shootout Notice'') to the other Shareholders offering either to purchase all their Shares or sell all the Initiating Shareholder's Shares at a specified price per Share.
    \item The Shootout Notice must specify the proposed price per Share, applicable payment terms, and completion date.
    \item The Recipient Shareholders shall have 20 Business Days to elect to either purchase all of the Initiating Shareholder's Shares or sell all of their Shares at the specified price.
    \item If the Recipient Shareholders fail to respond within the Response Period, they shall be deemed to have elected to sell all of their Shares to the Initiating Shareholder.
    \item Following the election, the purchasing Shareholder(s) shall be obligated to purchase, and the selling Shareholder(s) shall be obligated to sell, the relevant Shares at the specified price.
    \item If the purchasing Shareholder(s) fail to complete the purchase, the selling Shareholder(s) may either enforce the sale through legal proceedings or purchase the Shares of the defaulting Shareholder(s) at ninety percent (90\%) of the original price as a penalty.
    \end{enumerate}
\item As an alternative to the Texas Shootout procedure, the Shareholders may unanimously agree in writing to:
    \begin{enumerate}[label=(\roman*)]
    \item sell the entire Company to a third party;
    \item appoint an independent expert with casting vote authority on the specific matter causing the Deadlock; or
    \item implement any other agreed resolution mechanism.
    \end{enumerate}
\item The valuation of Shares for any transfer pursuant to this clause shall be the price specified in the Shootout Notice, unless the Shareholders unanimously agree to appoint an Independent Valuer to determine the Fair Market Value.
\item All parties shall cooperate fully with the procedures set out in this clause and shall execute all documents and take all actions necessary to give effect to any transfer of Shares resulting from these procedures.
\end{enumerate} 
\section{FINANCIAL PROVISIONS} \label{sec:financial_provisions}

\subsection{Financial Year and Accounting Records}
\begin{enumerate}[label=(\alph*)]
\item \textbf{Financial Year:} The financial year of the Company (``Financial Year'') shall end on 31 July each year, in accordance with the Shareholders Agreement, or as otherwise determined by the Board in accordance with the Corporations Act 2001.
\item \textbf{Accounting Records:} The Company shall cause proper, true, and complete accounting and other records to be kept, including:
    \begin{enumerate}[label=(\roman*)]
    \item Accurate and up-to-date financial records prepared in accordance with Australian Accounting Standards (AAS) and the Corporations Act 2001, consistently applied.
    \item Monthly management accounts, including a profit and loss statement, balance sheet, cash flow statement, and a comparison against the approved budget, to be provided to the Board and each Shareholder entitled to information rights under this Agreement within [e.g., fifteen (15)] Business Days of the end of each month.
    \item Annual financial statements, including a Director's report, which shall be audited if required by law, by this Agreement (see sub-clause \ref{subsec:Audit}), or by a resolution of Shareholders (as a Reserved Matter).
    \item All necessary taxation records and returns, including for Goods and Services Tax (GST), Pay As You Go (PAYG) withholding, income tax, and any other applicable taxes or levies.
    \item Detailed records of Research and Development (R\&D) expenditure, particularly any expenditure eligible for R\&D tax incentives or government grants, in a manner that facilitates such claims.
    \item A register of all material contracts, leases, and other significant commitments entered into by the Company.
    \end{enumerate}
\end{enumerate}

\subsection{Banking and Financial Controls}
\begin{enumerate}[label=(\alph*)]
\item \textbf{Bank Accounts:} All bank accounts of the Company shall be opened and maintained with such bank(s) as the Board may determine. Bank account mandates and authorities for payments and withdrawals shall be as determined by the Board from time to time, provided that any payments or series of related payments exceeding AUD 1,000 (or such other amount as the Board may agree unanimously and record in writing) shall require the signature or electronic approval of at least two (2) Directors, or one (1) Director and another senior person expressly authorised by the Board for this purpose.

\item \textbf{Expenditure Approval Thresholds:} The following expenditure approval requirements shall apply:
    \begin{enumerate}[label=(\roman*)]
    \item \textbf{Up to AUD 1,000:} Individual expenditures up to AUD 1,000 may be approved by any single Director or authorized management personnel within their delegated authority. This includes expense contracts (excluding customer revenue contracts) with total value up to AUD 1,000 and duration not exceeding 12 months.
    \item \textbf{Above AUD 1,000:} Any expenditure exceeding AUD 1,000, including expense contracts (excluding customer revenue contracts) with total value exceeding AUD 1,000 or duration longer than 12 months, requires unanimous Shareholder approval as a Reserved Matter.
    \item \textbf{Emergency Expenditures:} In genuine business emergency situations where immediate expenditure is required to prevent material harm to business operations, a single Director may authorize expenditure as follows:
    \begin{enumerate}[label=(\alph*)]
    \item \textbf{Standard Emergency Limit:} Up to AUD 5,000 for general business emergencies
    \item \textbf{Critical Business Emergency Limit:} Up to AUD 15,000 for critical situations including:
        \begin{enumerate}[label=(\roman*)]
        \item Payment of outstanding bills to prevent service disruption (utilities, internet, essential software)
        \item Emergency repairs to prevent business interruption
        \item Urgent legal or compliance requirements with statutory deadlines
        \item Critical system failures requiring immediate technical support
        \item Emergency travel for essential business purposes
        \end{enumerate}
    \item \textbf{Notification and Ratification:} Other Director must be notified within 24 hours and expenditure ratified at next Board meeting
    \item \textbf{Business Purpose Restriction:} Emergency expenditures must be strictly for business operations and cannot include personal expenses, non-essential purchases, or speculative investments
    \item \textbf{Documentation Requirement:} Full documentation of emergency nature and business necessity must be provided
    \end{enumerate}
    \item \textbf{Expense Contract Commitments:} For the avoidance of doubt, any expense contract or agreement that commits the Company to future expenditure (excluding customer revenue contracts) shall be treated as an expenditure equal to the total contract value for approval purposes, regardless of payment timing. Customer revenue contracts and agreements that generate income for the Company are excluded from these expenditure approval requirements.
    \end{enumerate}
\item \textbf{Internal Controls:} The Board shall establish, document, and maintain appropriate internal financial controls, systems, and procedures designed to safeguard the Company's assets, ensure the accuracy and reliability of its financial reporting, and promote operational efficiency and compliance with applicable laws and policies.
\item \textbf{Budget Adherence:} Management shall operate the Business within the scope of the annual budget approved by the Board and Shareholders (as a Reserved Matter). Any material deviation or proposed expenditure outside the approved budget or exceeding pre-agreed limits for specific categories shall require prior Board approval (and Shareholder approval if it triggers a Reserved Matter threshold).
\item \textbf{Annual Budget:} The Company shall prepare a detailed draft annual budget and business plan for each upcoming Financial Year. This draft shall be presented to the Board for review and refinement at least [e.g., forty-five (45)] days prior to the commencement of the Financial Year, and subsequently submitted to Shareholders for approval as a Reserved Matter at least [e.g., thirty (30)] days prior to the Financial Year's commencement.
\end{enumerate}

\subsection{Dividend Policy} \label{subsec:DividendPolicy}
\begin{enumerate}[label=(\alph*)]
\item The declaration and payment of any dividends by the Company shall be determined by the Board by unanimous resolution of all Directors, subject always to the Company satisfying the solvency test and other requirements of the Corporations Act 2001 immediately after any such payment.
\item In determining whether to declare or pay a dividend, and the amount of any such dividend, the Board shall have regard to (among other things):
    \begin{enumerate}[label=(\roman*)]
    \item The Company's available profits, retained earnings, and distributable reserves.
    \item The Company's current and anticipated future cash flow, working capital requirements, and capital expenditure needs.
    \item The Company's funding requirements for its approved business plan, including planned growth, R\&D activities (particularly in AI technology development and data acquisition), strategic initiatives, and expansion opportunities.
    \item Any restrictions or covenants under the Company's financing agreements or other contractual obligations.
    \item The general financial condition and performance of the Company.
    \item The desire to reinvest profits to enhance long-term Shareholder value and support sustainable growth.
    \item Any deferred payment obligations to exiting Shareholders as contemplated under this Agreement.
    \end{enumerate}
\item Unless otherwise agreed by Shareholders holding all issued Shares of a particular class, all dividends declared on a class of Shares shall be distributed to the holders of those Shares pro rata to their respective shareholdings of that class as at the record date fixed for that dividend.
\item The adoption of or any material alteration to the Company's general dividend policy shall be a Reserved Matter requiring Shareholder approval.
\end{enumerate}

\subsection{Audit} \label{subsec:Audit}
\begin{enumerate}[label=(\alph*)]
\item Unless the Shareholders unanimously agree otherwise in writing for a particular Financial Year and an audit is not otherwise required by the Corporations Act 2001 or any financing agreement, the Company's annual financial statements shall be audited by an independent registered company auditor (the ``Auditor'').
\item The Auditor shall be appointed (and may be removed) by the Shareholders by ordinary resolution (or as otherwise specified for the appointment/removal of auditors as a Reserved Matter in Schedule 2).
\item A copy of the audited annual financial statements, together with the Auditor's report thereon, shall be provided to each Shareholder within [e.g., one hundred and twenty (120)] days of the end of each Financial Year for which an audit is conducted.
\end{enumerate}

\subsection{Future Funding and Capital Contributions} \label{subsec:FutureFunding}
\begin{enumerate}[label=(\alph*)]
\item \textbf{Acknowledgement of Funding Needs:} The Shareholders acknowledge that the Company may require additional funding from time to time to finance its operations, growth, R\&D activities, and strategic objectives as outlined in its approved business plan.
\item \textbf{Initial Capital:} The initial capital of the Company has been contributed by the Shareholders through the subscription for their Initial Shares. \textit{(Note: If there were other initial loans or contributions, specify or reference them here or in a schedule).}
\item \textbf{Board Determination of Funding Requirement:} If the Board determines that the Company requires additional equity capital (``Additional Funding''), the Board shall prepare a proposal detailing the amount of Additional Funding required, the proposed use of funds, and an indicative valuation or price per share for the new equity (the ``Funding Proposal'').
\item \textbf{Offer to Existing Shareholders (Pre-emptive Right):}
    \begin{enumerate}[label=(\roman*)]
    \item Subject to sub-clause (e) regarding the investor pool commitment, any proposal for Additional Funding via the issue of new Shares by the Company must first be offered to all then-existing Shareholders pro rata to their existing percentage shareholdings (a ``Rights Offer'').
    \item The Company shall provide each Shareholder with a written Rights Offer notice, including the Funding Proposal, the subscription price per New Share (the ``Subscription Price''), the number of New Shares offered to that Shareholder, and an acceptance period of at least [e.g., fifteen (15)] Business Days (the ``Acceptance Period'').
    \item Shareholders wishing to subscribe for their entitlement (or a lesser amount) of New Shares must give written notice of acceptance to the Company before the expiry of the Acceptance Period and pay the relevant subscription monies by the specified date.
    \end{enumerate}
\item \textbf{Investor Pool Commitment:} The Company has authorized 1,190,476 Preference Shares (Non-Voting) (representing 10\% of the Company's total authorized share capital) for future third-party external investors. The terms of such investment (including valuation and timing) shall be a Reserved Matter. These shares will be issued as new shares from the authorized capital, ensuring that existing shareholders are diluted proportionally rather than through transfer of personal holdings.
\item \textbf{Shortfall from Rights Offer / Third Party Investment:}
    \begin{enumerate}[label=(\roman*)]
    \item If any Shareholder does not take up their full entitlement under a Rights Offer (a ``Shortfall''), the New Shares not taken up may first be offered to the other participating Shareholders pro rata to their acceptances (or as they may otherwise agree).
    \item If, after such an offer to participating Shareholders, a Shortfall still remains, or if the Additional Funding is to be raised (in whole or in part) from third-party investors (including from the Company's reserved investor pool), the Company may, subject to obtaining any Requisite Shareholder Approval for the issue of new shares or the terms of investment (as per Schedule 2), offer and issue New Shares or accept subscriptions from such third-party investors at a price per Share not less than the Subscription Price offered to existing Shareholders in any contemporaneous Rights Offer (unless otherwise approved as a Reserved Matter).
    \end{enumerate}
\item \textbf{Dilution for Non-Participation:} A Shareholder who does not subscribe for their full pro-rata entitlement of New Shares in a Rights Offer, or if New Shares are issued to third parties, will have their percentage shareholding in the Company diluted accordingly. No Shareholder shall have any claim against the Company or any other Shareholder as a result of such dilution arising from their own non-participation in a duly offered Rights Offer or an approved third-party investment.
\item \textbf{Determination of Subscription Price for New Shares:} Unless otherwise determined by Shareholders as a Reserved Matter, the Subscription Price for New Shares issued by the Company shall be determined by the Board, acting in good faith and in the best interests of the Company, based on a Fair Market Value assessment. Such assessment may consider a recent valuation by an Independent Valuer, recent arm's length transactions in the Company's Shares, a bona fide offer from a third-party investor, or other relevant methodologies appropriate for a private technology company.
\item \textbf{Shareholder Loans:} If agreed by the Board and the relevant Shareholder(s), Additional Funding may also be provided by way of Shareholder loans, on terms (including interest rate, security, and repayment) to be approved by the Board (and by Shareholders if it constitutes a Related Party Transaction requiring approval under Schedule 2).
\end{enumerate}

\subsection{Company Valuation for Internal Events} \label{subsec:CompanyValuationInternal}
\begin{enumerate}[label=(\alph*)]
\item For any provision in this Agreement that requires a determination of Fair Market Value (or similar term) for the Company's Shares in the context of an internal transfer, buy-out, or other event between Shareholders or between a Shareholder and the Company (e.g., certain leaver events, deadlock buy-sell mechanisms where a valuation is stipulated), and where a specific valuation mechanism for that event is not otherwise detailed, the following principles shall apply:
    \begin{enumerate}[label=(\roman*)]
    \item The Fair Market Value shall be determined by an Independent Valuer. ``Independent Valuer'' means a reputable independent chartered accountant, accredited business valuer, or investment bank with demonstrable experience in valuing private technology companies, particularly in the AI sector, who is not an Auditor of the Company at the time of valuation to avoid conflict (unless all parties agree otherwise).
    \item The Independent Valuer shall be appointed by mutual agreement of the affected parties (e.g., the selling Shareholder and the purchasing Shareholder(s)/Company). If they cannot agree on an Independent Valuer within [e.g., ten (10)] Business Days of one party proposing a valuer to the other, then any affected party may request the President of [e.g., the Australian Institute of Business Valuers or relevant accounting body] to appoint an Independent Valuer.
    \item The Independent Valuer shall be instructed to determine the Fair Market Value of the Shares in question on a going concern basis, assuming a hypothetical sale of 100\% of the Company between a willing buyer and a willing seller, without any discount for a minority interest or premium for a controlling interest applying to the specific parcel of Shares being valued, unless the specific clause in this Agreement requiring the valuation dictates otherwise.
    \item The Independent Valuer shall act as an expert and not as an arbitrator. Their valuation report, including reasoning, shall be provided to all affected parties and shall be final and binding on those parties, absent manifest error.
    \item The costs of the Independent Valuer shall be borne equally by the parties involved in the specific event requiring the valuation (e.g., selling Shareholder and purchasing Shareholder(s)/Company), unless this Agreement specifies otherwise for that particular event or the parties agree otherwise.
    \end{enumerate}
\end{enumerate}

\subsection{Insurance}
\begin{enumerate}[label=(\alph*)]
\item The Company shall maintain Directors and Officers liability insurance with minimum coverage of AUD 1,000,000.
\item The Company shall maintain appropriate business insurance including public liability and professional indemnity as determined by the Board.
\item Insurance costs are a legitimate business expense and shall be included in the Company's annual budget.
\item The Board shall review insurance coverage annually to ensure adequacy.
\end{enumerate}

// ... (Future Funding and Valuation subsections to follow) ... 
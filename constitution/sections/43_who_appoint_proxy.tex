\section{Who can appoint a proxy}

\begin{enumerate}[label=(\alph*)]
    \item A member who is entitled to attend and cast a vote at a meeting of the Company's members or at a meeting of the holders of a class of shares may appoint an individual or a body corporate as the member's proxy to attend and vote for the member at the meeting. The proxy need not be a member.
    
    \item The appointment may specify the proportion or number of votes that the proxy may exercise.
    
    \item If the member is entitled to cast two or more votes at the meeting, the member may appoint two proxies. If the member appoints two proxies and the appointment does not specify the proportion or number of the member's votes each proxy may exercise, each proxy may exercise half of the votes.
    
    \item Any fractions of votes resulting from the application of rule 43.b or rule 43.c are disregarded.
\end{enumerate}

\textbf{Note:} Proxy appointments are subject to the voting rights of each share class under the Shareholders Agreement. Only holders of Ordinary Shares (Voting) may appoint proxies for voting purposes, while holders of Preference Shares (Non-Voting) and Employee Ordinary Shares (Non-Voting) may only appoint proxies for matters specifically affecting their class. 
\section{Joint holders}

\begin{enumerate}[label=(\alph*)]
    \item Where two or more persons are registered as the holders of a share, they must be treated as holding the share as joint tenants with benefits of survivorship subject to rule 14.b and to the following:
    \begin{enumerate}[label=(\roman*)]
        \item the Company is not bound to register more than three persons (not being the trustees, executors or administrators of a deceased holder) as the holder of the share;
        \item the joint holders of the share are liable severally as well as jointly in respect of all payments which ought to be made in respect of the share;
        \item on the death of any one of the joint holders, the survivor or survivors are the only person or persons recognised by the Company as having any title to the share, but the directors may require such evidence of death as they see fit;
        \item any one of the joint holders may give effective receipts for any dividend, bonus or return of capital payable to the joint holders; and
        \item only the person whose name stands first in the Register as one of the joint holders of the share is entitled to delivery of the certificate relating to the share or to receive notices from the Company and a notice given to that person must be treated as notice to all the joint holders.
    \end{enumerate}
    
    \item Where three or more persons are registered holders of a share in the Register (or a request is made to register more than three persons) only the first three named persons are regarded as holders of the share and all other named persons must be disregarded for all purposes except in the case of executors or trustees of a deceased shareholder.
\end{enumerate}

\textbf{Note:} Joint holding arrangements must comply with the Shareholders Agreement, particularly regarding transfer restrictions and voting rights applicable to each share class. 
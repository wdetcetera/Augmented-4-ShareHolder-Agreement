\section{Dividends where different classes of shares}

\begin{enumerate}[label=(\alph*)]
    \item If there is more than one class of shares, any dividend whether interim or otherwise may be paid on the shares of any one or more class or classes to the exclusion of the shares of any other class or classes.
    
    \item If dividends are to be paid on more than one class, the dividend on the shares of one class may be at a higher or lower rate than one at the same rate as the dividend on the shares of another class, but the shares within each class must share equally in any dividend in respect of that class.
    
    \item An objection may not be raised to any resolution which:
    \begin{enumerate}[label=(\roman*)]
        \item determines a higher rate of dividend on the shares of any class than the dividend determined on the shares of any other class; or
        \item determines a dividend on the shares of any class to the exclusion of the shares of any other class,
    \end{enumerate}
    
    on the ground that:
    
    \begin{enumerate}[label=(\roman*)]
        \setcounter{enumii}{2}
        \item the resolution was passed by the votes of the holders of the shares of a class to receive the higher rate of dividend or to receive the dividend (as the case may be); and
        \item the resolution was not passed by the votes of the holders of the shares of a class not to receive the dividend or to receive the lower rate of dividend (as the case may be).
    \end{enumerate}
\end{enumerate} 
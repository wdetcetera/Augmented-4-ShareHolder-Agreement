\section{Form of proxy sent out by Company}

\begin{enumerate}[label=(\alph*)]
    \item A form of proxy sent out by the Company may be in a form determined by the directors but must:
    \begin{enumerate}[label=(\roman*)]
        \item enable the member to specify the manner in which the proxy must vote in respect of a particular resolution; and
        \item leave a blank for the member to fill in the name of the person primarily appointed as proxy.
    \end{enumerate}
    
    \item The form may provide that if the member leaves it blank as to the person primarily appointed as proxy or if the person or persons named as proxy fails or fail to attend, the chairperson of the meeting is appointed proxy.
    
    \item Despite rule 47.a an instrument appointing a proxy may be in the following form or in a form that is as similar to the following form as the circumstances allow:
    
    I/We, \underline{\hspace{3cm}} of \underline{\hspace{3cm}}, being a member/members of the abovenamed company, appoint \underline{\hspace{3cm}} of \underline{\hspace{3cm}}* \underline{\hspace{3cm}} of \underline{\hspace{3cm}} as my/our proxy to vote for me/us on my/our behalf at the *annual general/*general meeting of the company that will be held on \underline{\hspace{3cm}} at \underline{\hspace{3cm}} at \underline{\hspace{1cm}} and at any adjournment of that meeting. The proxy holder may vote as they think fit unless specified below:
    
    [table of resolutions and voting intentions]
    
    Signed On \underline{\hspace{3cm}}
    
    *The appointment may also include details of multiple proxies where the shareholder(s) has multiple voting shares, and also details of a substitute proxy in case the main proxy is unable to attend
\end{enumerate} 
\section{Transmission of shares}

\begin{enumerate}[label=(\alph*)]
    \item If a shareholder who does not own shares jointly dies, the Company will recognise only the personal representative of the deceased shareholder as being entitled to the deceased shareholder's interest in the shares.
    
    \item If the person entitled to shares as the personal representative of a deceased shareholder or because of the bankruptcy or mental incapacity of a shareholder (successor) gives the directors the information they reasonably require to establish the successor's entitlement to be registered as holder of the shares:
    \begin{enumerate}[label=(\roman*)]
        \item the successor may: (a) by giving a signed notice to the Company, elect to be registered as the holder of the shares; or (b) by giving a completed transfer form to the Company, transfer the shares to another person; and
        \item the successor, whether or not registered as the holder of the shares, is entitled to the same rights, and is subject to the same liabilities, as if the successor were registered as holder of the shares.
    \end{enumerate}
    
    \item On receiving an election under rule 22.b.i.a, the Company must register the successor as the holder of the shares.
    
    \item A transfer under rule 22.b.i.b is subject to the same rules (for example, about entitlement to transfer and registration of transfers) as apply to transfers generally.
    
    \item If a shareholder who owns shares jointly dies, the Company will recognise only the survivor as being entitled to the deceased shareholder's interest in the shares. The estate of the deceased shareholder is not released from any liability in respect of the shares.
    
    \item This rule 22 has effect subject to the Bankruptcy Act 1966.
\end{enumerate}

\textbf{Note:} Transmission of shares upon death, bankruptcy, or incapacity must comply with the Shareholders Agreement provisions, including any specific procedures for different share classes and potential triggering of buyback or transfer rights. 
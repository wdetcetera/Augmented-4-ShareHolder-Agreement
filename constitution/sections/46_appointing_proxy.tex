\section{Appointing a proxy}

\begin{enumerate}[label=(\alph*)]
    \item An appointment of a proxy is valid if it is signed or otherwise electronically authenticated (as referred to in the Corporations Regulations 2001, and in rules 46.b and 46.c) by the member making the appointment and contains the following information:
    \begin{enumerate}[label=(\roman*)]
        \item the member's name and address;
        \item the Company's name;
        \item the proxy's name or the name of the office held by the proxy; and
        \item the meetings at which the appointment may be used.
    \end{enumerate}
    
    An appointment may be a standing one.
    
    \item An electronically authenticated appointment of a proxy must in addition to rule 46.a:
    \begin{enumerate}[label=(\roman*)]
        \item include a method of identifying the member; and
        \item include an indication of the member's approval of the information communicated.
    \end{enumerate}
    
    \item If the electronically authenticated appointment of a proxy is done through either email or internet-based voting:
    \begin{enumerate}[label=(\roman*)]
        \item the member must be identified by personal details such as the member's name, personal address and date of birth; and
        \item the member's approval must be communicated by a form of security protection (for example, the entering of a confidential identification number such as a shareholder registration number or holder identification number).
    \end{enumerate}
    
    \item An undated appointment is taken to have been dated on the day it is given to the Company.
    
    \item An appointment may specify the way the proxy is to vote on a particular resolution. If it does:
    \begin{enumerate}[label=(\roman*)]
        \item the proxy need not vote on a show of hands, but if the proxy does so, the proxy must vote that way;
        \item if the proxy has two or more appointments that specify different ways to vote on the resolution – the proxy must not vote on a show of hands;
        \item if the proxy is the chairperson – the proxy must vote on a poll, and must vote that way; and
        \item if the proxy is not the chairperson – the proxy need not vote on a poll, but if the proxy does so, the proxy must vote that way.
    \end{enumerate}
    
    If a proxy is also a member, this rule 46.e does not affect the way that the person can cast any votes the person holds as a member.
    
    \item An appointment does not have to be witnessed.
    
    \item A later appointment revokes an earlier one if both appointments could not be validly exercised at the meeting.
    
    \item If a share is held jointly an appointment of proxy may be signed by any one of the joint holders, but if the Company receives more than one appointment for the same share:
    \begin{enumerate}[label=(\roman*)]
        \item an appointment signed by all the joint holders is accepted in preference to an appointment signed by the member whose name appears first in the Register or by any other member holding the share jointly; and
        \item subject to rule 46.h.i an appointment signed by the member whose name appears first in the Register is accepted in preference to an appointment signed by any other member or members holding the share jointly.
    \end{enumerate}
\end{enumerate} 
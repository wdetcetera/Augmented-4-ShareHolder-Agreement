\section{Shareholders' rights on winding up}

\begin{enumerate}[label=(\alph*)]
    \item Subject to this constitution and the rights or restrictions attached to any shares or class of shares:
    \begin{enumerate}[label=(\roman*)]
        \item if the Company is wound up and the property of the Company available for distribution among the members is more than sufficient to pay:
        \begin{enumerate}[label=(\Alph*)]
            \item all the debts and liabilities of the Company; and
            \item the costs, charges and expenses of the winding up,
        \end{enumerate}
        
        the excess must be divided among the members in proportion to the number of shares held by them, irrespective of the amounts paid or credited as paid on the shares;
        
        \item for the purpose of calculating the excess referred to in rule 137.a.i, any amount unpaid on a share is to be treated as property of the Company;
        
        \item the amount of the excess that would otherwise be distributed to the holder of a partly paid share under rule 137.a.i must be reduced by the amount unpaid on that share at the date of the distribution; and
        
        \item if the effect of the reduction under rule 137.a.iii would be to reduce the distribution to the holder of a partly paid share to a negative amount, the holder must contribute that amount to the Company.
    \end{enumerate}
    
    \item If the Company is wound up, the liquidator may, with the sanction of a special resolution, divide among the members in kind the whole or any part of the property of the Company and may for that purpose set the value the liquidator considers fair upon any property to be so divided and may determine how the division is to be carried out as between the members or different classes of members. This division need not be in accordance with the legal rights of the members, and in particular, any class may be given preferential or special rights or may be excluded altogether or in part.
    
    \item The liquidator may, with the sanction of a special resolution, vest the whole or any part of the property referred to in rule 137.a in trustees upon trusts for the benefit of the contributories that the liquidator sees fit, but so that no member is compelled to accept any shares or other securities on which there is any liability.
\end{enumerate} 
\section{Variation of class rights}

\begin{enumerate}[label=(\alph*)]
    \item Rights attached to shares in a class of shares may be varied or cancelled only:
    \begin{enumerate}[label=(\roman*)]
        \item by special resolution of the Company; and
        \item either: (a) by special resolution passed at a meeting of the members holding shares in the class; or (b) with the written consent of members with at least 75\% of the votes in the class.
    \end{enumerate}
    
    \item Rule 25.a applies whether or not the Company is being wound up.
    
    \item The Company must give a notice in writing of the variation or cancellation of shares to members of the class affected within seven days after the variation or cancellation.
    
    \item The provisions of this constitution relating to general meetings apply so far as they are capable of application and with the necessary changes to every meeting of members holding shares in a class except that:
    \begin{enumerate}[label=(\roman*)]
        \item a quorum is constituted by not less than two members (personally present or represented by a duly appointed proxy, attorney or representative) holding at least 25\% of the issued shares of the class or if there is one holder of shares in a class, that person; and
        \item any member who holds or represents shares of the class may demand a poll; and
        \item the Auditor is not entitled to notice of the meeting or to attend or speak at the meeting.
    \end{enumerate}
\end{enumerate}

\textbf{Note:} Any variation of class rights must comply with the Shareholders Agreement, particularly the specific rights and protections established for Ordinary Shares (Voting), Preference Shares (Non-Voting), and Employee Ordinary Shares (Non-Voting). Certain variations may constitute Reserved Matters requiring unanimous founder approval. 
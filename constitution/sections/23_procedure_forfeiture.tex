\section{Procedure for forfeiture}

\begin{enumerate}[label=(\alph*)]
    \item If a member fails to pay a call or instalment of a call on the day appointed for payment of the call or instalment or fails to pay any money payable under rule 10 the directors may while any part of the call or instalment or other money remains unpaid serve a notice on the member requiring payment of so much of the call or instalment or other money as is unpaid together with any interest that has accrued and the costs, expenses or damages that the Company has incurred due to the failure to pay.
    
    \item The notice must:
    \begin{enumerate}[label=(\roman*)]
        \item appoint a further date (not earlier than the expiration of 14 days after the date of service of the notice) on or before which the payment required by the notice is to be made; and
        \item state that, in the event of nonpayment at or before the further day appointed, the shares in respect of which the call was made will be liable to be forfeited.
    \end{enumerate}
    
    \item If the requirements of a notice served under rule 23.a are not complied with, any share in respect of which the notice has been given may, unless the payment required by the notice has been made, be forfeited by a resolution of the directors to that effect.
    
    \item The forfeiture includes all dividends determined or payable in respect of the forfeited share and not actually paid before the forfeiture.
    
    \item The Company may sell a forfeited share or otherwise dispose of it on terms and in a manner the directors see fit.
    
    \item The directors may at any time before a forfeited share has been sold or otherwise disposed of, annul the forfeiture upon conditions they see fit.
    
    \item A person whose shares have been forfeited ceases to be a member in respect of the forfeited shares, but (unless the ordinary shareholders resolve otherwise) remains liable to pay and must immediately pay to the Company all calls, instalments, interest and expenses owing on or payable in respect of the shares at the time of forfeiture together with interest from the time of forfeiture until payment at the rate determined by the directors. The directors may enforce payment of the money as they see fit but are not under any obligation to do so.
    
    \item A statement in writing declaring that the person making the statement is a director or a secretary of the Company, and that a share in the Company has been duly forfeited on a date stated is prima facie evidence of the facts as against all persons claiming to be entitled to the share.
    
    \item The provisions of this constitution as to forfeiture apply in the case of nonpayment of any sum that, by the terms of issue of a share, becomes payable at a fixed time, as if that sum had been payable by virtue of a call duly made and notified.
\end{enumerate}

\textbf{Note:} Any forfeiture of shares must comply with the Shareholders Agreement, particularly regarding the different procedures and restrictions applicable to each share class. Employee Ordinary Shares (Non-Voting) may have specific forfeiture provisions under employment-based vesting schedules. 
\section{Right to lien}

\begin{enumerate}[label=(\alph*)]
    \item Subject to rule 9.d, the Company has a first and paramount lien on all shares registered in the name of a member (whether solely or jointly with others) for all money presently payable by the member or the member's estate to the Company.
    
    \item The directors may at any time exempt a share wholly or in part from the provisions of this rule 9.
    
    \item The Company's lien (if any) on a share extends to all dividends payable in respect of the share. The directors may retain those dividends and apply them in or towards satisfaction of all amounts due to the Company in respect of which the lien exists.
    
    \item The amount of the Company's lien is restricted to:
    \begin{enumerate}[label=(\roman*)]
        \item unpaid calls and instalments upon the specific shares in respect of which calls or instalments are due and unpaid;
        \item if the shares were acquired under an employee incentive scheme an amount owed to the Company for acquiring them; and
        \item an amount that the Company is required by law to pay (and has paid) in respect of the shares of a member or deceased former member.
    \end{enumerate}
    
    \item The Company's lien on a share extends to reasonable interest and expenses incurred because an amount referred to in rule 9.d is not paid.
    
    \item Unless otherwise agreed the registration of a transfer document operates as a waiver of the Company's lien (if any) on the shares transferred.
\end{enumerate}

\textbf{Note:} The exercise of any lien rights under this section is subject to the transfer restrictions and procedures set out in the Shareholders Agreement, particularly regarding Employee Ordinary Shares (Non-Voting) acquired under employee incentive schemes. 
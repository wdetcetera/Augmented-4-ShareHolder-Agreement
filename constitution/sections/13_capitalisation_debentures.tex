\section{Power to capitalise and issue debentures to members}

\begin{enumerate}[label=(\alph*)]
    \item The Company may capitalise profits. The capitalisation need not be accompanied by the issue of shares.
    
    \item The directors, or the Company in general meeting on the recommendation of the directors, may apply profits, including reserves and sums otherwise available for distribution to members, to:
    \begin{enumerate}[label=(\roman*)]
        \item pay up any amount unpaid on shares;
        \item issue shares, debentures or unsecured notes to members credited as fully paid up; or
        \item partly as mentioned in rule 13.b.i and partly as mentioned in rule 13.b.ii.
    \end{enumerate}
    
    \item The amount applied under rule 13.b must be applied for the benefit of members in the proportions in which the members are entitled to dividends.
    
    \item For the purpose of rule 13.c the directors may to the extent necessary to adjust the rights of the members among themselves:
    \begin{enumerate}[label=(\roman*)]
        \item issue fractional certificates or make cash payments in cases where shares, debentures or unsecured notes become issuable in fractions;
        \item determine the amount payable to a member under rule 13.b if there is no proportional entitlement;
        \item fix the value for distribution of any specific assets or any part of them;
        \item round down any payment to the nearest dollar; and
        \item vest any cash or specific assets in trustees upon trust for the persons entitled to the dividend or capitalised fund.
    \end{enumerate}
\end{enumerate}

\textbf{Note:} Any capitalisation or distribution under this section must comply with the Shareholders Agreement, particularly regarding the different dividend and distribution rights of the various share classes (Ordinary Shares (Voting), Preference Shares (Non-Voting), and Employee Ordinary Shares (Non-Voting)), including the 8\% cumulative preference dividend rights and liquidation preferences. 